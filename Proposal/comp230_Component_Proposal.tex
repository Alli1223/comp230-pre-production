% Please do not change the document class
\documentclass{scrartcl}

% Please do not change these packages
\usepackage[hidelinks]{hyperref}
\usepackage[none]{hyphenat}
\usepackage{setspace}
\doublespace

% You may add additional packages here
\usepackage{amsmath}
\usepackage{graphicx}
\usepackage{wrapfig}


% Please include a clear, concise, and descriptive title
\title{Comp230 Component Proposal} 

% Please do not change the subtitle
%\subtitle{Comp230_Component_Proposal}

% Please put your student number in the author field
\author{1507516}

\begin{document}

\maketitle

\abstract{(possibly subject to change based on the BA game choice)}



\section{Multiplayer component Proposal}

\subsection{What is the product?}
The product I intend to make is a networking component that will make the BA game multiplayer.
This will allow players to play together over the internet or local network.

Is there a market? Who is the target audience?
There is currently a large market for online multiplayer games at the moment, as most of the new AAA titles feature some sort of multiplayer interaction. 

Also according to one article I found online, adding PVP multiplayer to your game could increase revenue by 510%.
http://www.gamedonia.com/blog/pvp-increase-game-revenue

The target Audience would be people who enjoy playing games with their friends online, or at LAN parties.

\subsection{What are the unique selling points?}

Depending on the type of game the BA choose to make, there is a lot of possibilities for unique selling points.

 Could have some form of team combo mechanics, where the players can combine abilities.

\subsection{Is the scope appropriate for the product development time-frame?}

I believe with using the unity engine and possibly some third-party software this will be achievable. I can always cut back the scope of the multiplayer interactions if the scope is too large. 


\bibliographystyle{ieeetr}


\end{document}
