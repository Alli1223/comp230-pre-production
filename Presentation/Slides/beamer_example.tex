\documentclass{beamer}
\usepackage{etoolbox}\newtoggle{printable}\togglefalse{printable}
\usetheme{Copenhagen}
\usepackage{listings}
\usepackage{algpseudocode}
\pdfmapfile{+sansmathaccent.map}

\title{Comp230 Networking component}
\author{Alastair Rayner}
\date{\today}

\begin{document}

\maketitle

\begin{frame}{About the Networking Component}
	\begin{itemize}
		\item For my comp230 collaborative project I chose to work with a BA team and help them with the networking component of their game.\pause
		\item Their game is being built in unity. \pause
		\item Unity has quite a high level networking architecture already built in it.  \pause
		\item I have been working on a lobby GUI which was based of a Unity lobby template. \pause
		\item As well as helping the programmers in the BA team integrate networked movement and health.
	\end{itemize}
\end{frame}

\begin{frame}{The Market and Target Audience}
	\begin{itemize}
		\item There is a large audience when it comes to multilayer games. \pause
		\item The target audience would be people who like to play video games with their friends, especially at LAN parties etc. \pause
		\item Point 3 \pause
			\begin{itemize}
				\item Sub-Point 1 
				\item Sub-Point 2 \pause
				\item Sub-Point 3 
				\item Sub-Point 4 \pause
				\item Sub-Point 5 
				\item Sub-Point 6 \pause
			\end{itemize}
		\item Point 4 \pause
		\item Point 5 \pause
		\item Point 6
	\end{itemize}
\end{frame}

\begin{frame}{Images of the Lobby GUI}
	\begin{itemize}
		\item Lobby Menu \pause

	\end{itemize}
\end{frame}


\begin{frame}{Multiple columns!}
	\begin{columns}
		\begin{column}{0.27\textwidth}
			\begin{itemize}
				\item Point 1 \pause
				\item Point 2 \pause
				\item Point 3 \pause
				\item Point 4 \pause
				\item Point 5 \pause
				\item Point 6
			\end{itemize}
		\end{column}
		\begin{column}{0.67\textwidth}
			\begin{itemize}
				\item Point 1 \pause
				\item Point 2 \pause
				\item Point 3 \pause
				\item Point 4 \pause
				\item Point 5 \pause
				\item Point 6
			\end{itemize}
		\end{column}
	\end{columns}
\end{frame}


\begin{frame}
	\begin{center}
		\url{https://www.youtube.com/watch?v=7OtW-LLZ2OA}
	\end{center}
\end{frame}

\begin{frame}[fragile]{Code listing}
	\begin{lstlisting}[language=Python]
		a = 0
		for i in xrange(10):
			a += i

		print a
	\end{lstlisting}
\end{frame}

\begin{frame}{Pseudocode}

	{\tiny
	\begin{algorithmic}[1]
		\Procedure{MyProcedure}{}
			\State $\textit{stringlen} \gets \text{length of }\textit{string}$
			\State $i \gets \textit{patlen}$ \pause
			\State \emph{top}:
			\If {$i > \textit{stringlen}$} \Return false
			\EndIf
			\State $j \gets \textit{patlen}$ \pause
			\State \emph{loop}:
			\If {$\textit{string}(i) = \textit{path}(j)$}
				\State $j \gets j-1$.
				\State $i \gets i-1$.
				\State \textbf{goto} \emph{loop}.
				\State \textbf{close};
			\EndIf \pause
			\State $i \gets i+\max(\textit{delta}_1(\textit{string}(i)),\textit{delta}_2(j))$.
			\State \textbf{goto} \emph{top}.
		\EndProcedure
	\end{algorithmic}
	}
\end{frame}

\end{document}
